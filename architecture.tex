\section{Arquitetura de sistema}

A arquitetura do sistema desenvolvido pela equipe Oshkosh pertence a classe de
arquitetura de camadas, onde cada camada depende dos dados das camadas
inferiores, não das superiores. Apesar de citado como característica comum
em sistemas de camadas, no artigo do grupo \citep{chen2009terramax}, as
dependências não necessariamente são de cima para baixo. A arquitetura de três
camadas, estado da arte das arquiteturas robóticas, utilizada por diversos
sistemas autônomos, como o AUV Phoenix \citep{gat1991reliable} e o veículo
Stanley \citep{montemerlo2006winning}, pode possuir dependências no estilo
``top-down'' ou interdependência entre camadas.

Essa arquitetura apresenta a grande vantagem de ser flexível e
modular, já que o desenvolvimento de cada camada não depende das outras e as
dependências podem ser simuladas através de objetos \textit{mock}. Com isso,
cada equipe pode trabalhar em sua especialidade, otimizando o tempo e
aumentando a qualidade dos algoritmos desenvolvidos em cada função. Vale
ressaltar que, ao fim do desenvolvimento de cada subsistema, o sistema deverá
ser integrado, portanto o protocolo de comunicação entre camadas e a
plataforma/framework de integração é crucial nesse tipo de arquitetura.

São apresentados sete sistemas~\ref{fig:camadas}: 

\begin{enumerate}
  \item \textit{System Control and User Interface} - normalmente encontrado na
  literatura como \textit{Mission Control System}, ou planejamento de
missão (\textit{Mission Planning}) ou planejamento de tarefas (\textit{Task
Planning}). Como definidpo em \cite{fryxell1996navigation}, o controle de
missão é um sistema que permite ao operador definir as missões de um veículo em
linguagem de alto nível, provê ferramentas adequadas para converter planos em
Programas de Missões que podem ser verificados e executados em tempo real, e
permite ao operador saber o estado da missão enquanto esta é executada, e
modificá-la se for necessário.

  \item \textit{Autonomous Behavior} - é a parte que compõe a autonomia do robô,
  responsável pelo planejamento de trajetórias, e gerenciamento dos
  comportamentos do robô.
  
  \item \textit{Vehicle Management} - camada com os algoritmos de controle
  (controle de posição, estabilidade, velocidade,e  outros).
  
  \item \textit{Perception} - camada que possui os algoritmos de percepção, como
  detecção de obstáculos, detecção da via, detecção de tráfego, detecção de
  faixas, e outros.
  
  \item \textit{Sensor} - \textit{drivers} de sensores, usados para os
  algorítmos de percepção, como câmeras, LIDAR, e outros.
  
  \item \textit{Vehicle Status and Navigation} - sensores para o sistema de
  controle, como posição, velocidade, aceleração e estado do veículo.
\end{enumerate}
 
 \begin{figure}[!ht]
\centering
\includegraphics[width=\columnwidth]{figs/camadas.png}
\caption{Arquitetura de camadas do veículo TerraMax}
\label{fig:camadas}
\end{figure}
 
