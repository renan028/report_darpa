\section{Conclusões}

O TerraMax foi o único veículo realmente voltado a aplicações militares de logística dentre os participantes do desafio financiado pela agência de defesa americana. O sistema obteve resultados relativamente satisfatórios durante a fase de desenvolvimento e mesmo durante a competição, sendo capaz de realizar várias manobras complexas autonomamente.

No entanto, dados os problemas apresentados durante as etapas de teste e a performance no desafio principal, ficaram evidentes algumas limitações do robô frente a situações não previstas, como já apontado neste relatório. A lição aprendida é que robustez e resiliência são essenciais em sistemas autônomos.

Apesar de não ter completado a prova no evento final do DARPA Grand Challenge 2007, o TerraMax foi de fundamental importância para impulsionar o desenvolvimento de tecnologias de robótica autônoma, principalmente voltada à aplicação militar.
