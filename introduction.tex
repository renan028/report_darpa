\section{Introdução}

O TerraMax é um veículo militar autônomo precursor de diversas tecnologias de autonomia voltada para veículos militares terrestres existentes hoje em dia. Desenvolvido pela equipe Oshkosh, o veículo participou dos desafios da DARPA (\emph{Defense Advanced Research Projects Agency}) em 2004, 2005 e 2007.

Este documento relata o desenvolvimento do veículo autônomo, com foco no desafio de 2007, o DARPA Urban Challenge, no qual o robô deve trafegar em vias urbanas respeitando leis de trânsito e cumprindo diversas manobras desafiadoras sem interferência humana.

Nesta seção introdutória, serão apresentados um breve histórico do robô, as motivações e objetivos por trás de seu desenvolvimento, a equipe desenvolvedora, e uma visão geral do veículo e o \emph{hardware} utilizado.

\subsection{Motivações e Objetivos}

O TerraMax é o único veículo, dentre todos os participantes dos DARPA \emph{Grand Challenges} (DGC) de 2004, 2005 e 2007, voltado para a aplicação militar. O objetivo da equipe Oshkosh era desenvolver tecnologia de robótica autônoma para aplicação em veículos militares terrestres com a motivação de aprimorar o suporte logístico de cargas pesadas em campos de batalha.

Tendo isso em mente, a equipe desenvolveu o robô de modo a não somente satisfazer critérios de performance nos desafios como também torná-lo apto à operação em campo. Desse modo, a escolha do \emph{hardware} e o desenvolvimento dos algoritmos de navegação considerou a aplicação militar na prática. Por exemplo, os computadores embarcados selecionados são robustos, para condições "ultra-harsh", com encapsulamentos com certificações IP, resistência a impacto e vibração, entre outros.

Outro objetivo da equipe era desenvolver um sistema modular, de forma que novas funcionalidades poderiam ser incluídas no veículo com facilidade, e que o mesmo sistema pudesse ser utilizado para diferentes veículos militares.

\subsection{Histórico}

O TerraMax participou do desafio de 2004, em que nenhuma equipe conseguiu completar os objetivos da prova, e do DGC \emph{off-road} de 2005, antes de sua participação no DGC urbano de 2007.

A equipe realizou mudanças significativas de 2004 para 2005, como a direção das rodas traseiras para melhor desempenho em curvas acentuadas, e a completa reformulação do sistema autônomo. O TerraMax completou o desafio \emph{off-road} na quinta posição, após 28h ininterruptas de operação com sucesso em desvio de obstáculos e navegação em diferentes situações de estrada.

Para o desafio urbano, a equipe incluiu novos sensores (câmeras e LIDARs), sobretudo nas laterais e na traseira do veículo, de modo a adequar o sistema para navegação urbana. No ambiente \emph{off-road} do DCG 2005, não era necessário o sensoriamento lateral e traseiro, o que é de extrema importância no ambiente urbano para que o veículo lide com ultrapassagens e cruzamentos.

\subsection{Equipe do DGC 2007}

\subsection{Veículo e \emph{Hardware}}
